\documentclass[twocolumn,fleqn,a4paper,10pt]{jarticle}
\usepackage{amsmath,amssymb}
\usepackage[dvipdfmx]{graphicx}

\title{微分積分II春課題} 
\date{\today} 

\hoffset = -40pt
\voffset = -110pt
\textwidth = 180mm
\textheight = 275mm

\begin{document}
\setlength{\parindent}{0pt}
\setlength{\columnseprule}{0.4pt}
\setlength{\mathindent}{0pt}

\renewcommand{\thesection}{\fbox{\arabic{section}}}
\renewcommand{\labelenumi}{(\theenumi)}

\maketitle

%1
\section{}
\begin{enumerate}
\item \begin{flalign*}
	&\frac{x+5}{(x+2)(x-1)}=\frac{a}{x+2}+\frac{b}{x-1}\\
	&x+5=(x-1)a+(x+2)b\\
	&x=-2\text{とすると}a=-1\\
	&x=1\text{とすると}b=2
\end{flalign*}
\item \begin{flalign*}
	&\frac{1}{(x-3)(x-1)}=\frac{a}{x-3}+\frac{b}{x-1}\\
	&1=(x-1)a+(x-3)b\\
	&x=3\text{とすると}a=\frac{1}{2}\\
	&x=1\text{とすると}b=-\frac{1}{2}
\end{flalign*}
\item \begin{flalign*}
	&\frac{2x}{(x-1)(x^2+1)}=\frac{a}{x-1}+\frac{bx+c}{x^2+1}\\
	2x&=(x^2+1)a+(x-1)(bx+c)\\
	&x=1\text{とすると}a=1\\
	2x&=ax^2+a+bx^2+cx-bx-c\\
	&=(a+b)x^2+(c-b)x+a-c\\
	&a=1\text{だから}\\
	2x&=(1+b)x^2+(c-b)x+1-c\\
	&\text{係数を比較して}\\
	&a=1,b=-1,c=1
\end{flalign*}
\end{enumerate}
\end{document}