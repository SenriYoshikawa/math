\documentclass[a4paper,10pt]{jarticle}
\usepackage{amsmath,amssymb}
\usepackage[dvipdfmx]{graphicx}
%\usepackage{emathT}

\title{解析学A期末試験} 
\date{\today} 

\hoffset = -60pt
\voffset = -110pt
\textwidth = 150mm
\textheight = 275mm

\begin{document}
\setlength{\parindent}{0pt}
\setlength{\columnseprule}{0.4pt}

\renewcommand{\thesection}{\fbox{\arabic{section}}}
\renewcommand{\labelenumi}{(\theenumi)}

\maketitle

%1
\section{後期中間試験で$f(x,y)=x^3-3x(1+y^2)$が極値を持たないことを示した。このことを認めて次の各問いに答えよ。}
\begin{enumerate}
\item 単位円周$C=\{(x,y) \in \mathcal{R}^2 | \phi(x,y)=x^2+y^2=1\}$上における$f(x,y)$の最大値を求めよ。C上で最大値を持つことは認めてよい
\begin{gather*}
2\sqrt{2}
\end{gather*}
\item 単位閉円版$D=\{(x,y) \leq R^2|x^2+y^2 \leq 1\}$上での$f(x,y)$の最大値を求めよ。$D$上で最大値を持つことは認めてよい。
\begin{gather*}
\end{gather*}
\end{enumerate}

%2
\section{}
\begin{enumerate}
\item$D=\{(x,y)\in \mathcal{R}^2|0 \leq y \leq \frac{\pi}{4}, 0 \leq x \leq \sin{y}\}$を図示せよ。
\begin{gather*}
\end{gather*}
\item $I=\iint_{D} \sqrt{1-x^2} dxdy$を求めよ。
ただし公式$\int\sqrt{1-x^2}dx=\frac{1}{2}(x\sqrt{1-x^2}+\arcsin{x})+c$を用いてよい
\begin{gather*}
\frac{1}{8}+\frac{\pi^2}{16}
\end{gather*}
\end{enumerate}

%3
\section{}
\begin{enumerate}
\item $D=\{(x,y) \in \mathcal{R}^2 | 0 \leq x-y \leq 1, 0 \leq x+y \leq 1\}$を図示せよ。なお(2)で変数変換を用いるときは新たな積分領域も描け。
\begin{gather*}
\end{gather*}
\item$ I = \iint_{D} (x-y)\log{(x+y+1)}dxdy $を求めよ
\begin{gather*}
\frac{1}{4}(2\log{2}-1)
\end{gather*}
\end{enumerate}

%4
\section{}
\begin{enumerate}
\item $ D=\{(x,y) \leq \in \mathcal{R}^2 | x^2+y^2 \leq x\}$を図示せよ。なお(2)で変数変換を用いる場合は新たな積分領域も描け。
\begin{gather*}
\end{gather*}
\item$ I=\iint_{D} \sqrt{1-x^2-y^2}dxdy$を求めよ
\begin{gather*}
\frac{\pi}{3}-\frac{4}{9}
\end{gather*}
\end{enumerate}

\end{document}




























